\documentclass[12pt]{article}
\usepackage{amsmath, amsthm, amssymb, amsfonts, graphicx}

\title{Exam II \\
\small Differential Equations \\
University of Massachusetts Lowell \\
Summer 2020}
\author{Joel Savitz}

\newcommand{\reals}{\mathbb{R}}
\newcommand{\cplx}{\mathbb{C}}

\begin{document}
\maketitle

\textbf{1. Solve the IVP:}

Given:

\begin{align}
	\label{g1.1}
	y'' - 5y' -6 = 0 \\
	\label{g1.2}
	y(0) = 4 \land y'(0) = 3
\end{align}

Guess that $y = e^{rx}$ for some $r \in \reals$, then:
\begin{align}
	\label{p1.3}
	y' = re^{rx} \land y'' = r^2e^{rx} & \\
	r^2e^{rx}-5re^{rx}-6e^{rx} = 0 & \textrm{ (apply (\ref{p1.3}) to (\ref{g1.1})}) \\
	r^2-5r-6 = 0 & \textrm{ (divide by $e^{rx}$)} \\
	\label{p1.7}
	(r-6)(r-1) = 0 \implies r \in \{ -1, 6 \} & \\
	\label{p1.8}
	(\ref{p1.7}) \implies y(x) = Ae^{-x} + Be^{6x} \\
	(\ref{g1.2}) \land (\ref{p1.8}) \implies A + B = 4 & \textrm{ (apply initial 1)}\\
	\label{p1.9}
	(\ref{p1.8}) \implies \frac{dy}{dx} = -Ae^{rx} + 6Be^{rx} \\
	(\ref{g1.2}) \land (\ref{p1.9}) \implies -A + 6B = 3 & \textrm{ (apply initial 2)}\\
	\Big(\begin{bmatrix} 1 & 1 & 4 \\ -1 & 6 & 3 \end{bmatrix} \sim
	\begin{bmatrix} 1 & 0 & 3 \\ 0 & 1 & 1 \end{bmatrix} \Big)
	\iff \Big(A = 3 \land B = 1 \Big) \\
	\label{s1}
	\therefore y(x) = 3e^{-x} + e^{6x}
\end{align}

\medskip

\textbf{2. Find and categorize the equilibrium solutions of the ODE}

Given:

\begin{align}
	\label{g2}
	\frac{dy}{dx} = y(y - 1)^2
\end{align}

Let $f(y) = \frac{dx}{dy}$, then we have table \ref{t1}.

We see that $f$ is decreasing at values approaching $y = 0$ from the negative side
and increasing at values approacing $y = 0$ from the positive side,
so we say that the equilibrium solution $y = 0$ is unstable.

We also see that $f$ is increasing at values approaching $y = 1$ from both sides,
so we say that the equilibrium solution $y = 1$ is semi-stable.


\begin{table}
\begin{tabular}{l|l}
$y$                              & $f(y)$                           \\ \hline
$-3$                             & $-48$                            \\
$-2$                             & $-18$                            \\
$-1$                             & $-4$                             \\
$0$                              & $0$                              \\
$\frac{1}{2}$ 			 & $\frac{1}{8}$ 		    \\
$1$                              & $0$                              \\
$2$                              & $2$                              \\
$3$                              & $12$                            
\end{tabular}
\centering
\label{t1}
\caption{The rate of change of the dependent variable at different values}
\end{table}

\medskip

\textbf{3. Find the general solution}

Given:

\begin{align}
	\label{g3.1}
	y = x^r & \textrm{ for some $r \in \reals$} \\
	\label{g3.2}
	xy'' - 6y' = 0
\end{align}

We can use this information to find the general solution:

\begin{align}
	\label{p3.3}
	y' = rx^{r-1} \land y'' = (r^2-r)x^{r-2} \\
	(\ref{g3.2}) \land (\ref{p3.3}) \implies (r^2-r)x^{r-1} - 6rx^{r-1} = 0 \\
	\label{p3.4}
	r^2-7r = r(r-7) = 0 \textrm{ (divide by $x^{r-1}$)} \\
	(\ref{p3.4}) \implies r \in \{ 0, 7 \} \land y(x) = Ax^0 + Bx^7 \textrm{ for any } A,B \in \reals \\
	\therefore y(x) = A + Bx^7 \textrm{ for any real $A$ and $B$}
\end{align}

\medskip

\textbf{4. Population over time}

Suppose we have a population of hamsters or something described by the following equations,
where 2010 is considered to be the begining of time.

\begin{align}
	\label{g4.1}
	\frac{dP}{dt} = AP \textrm{ for some } A \in \reals \\
	\label{g4.2}
	P(0) = 100 \land P(2) = 400
\end{align}

We see that equation \ref{g4.2} is seperable
and so we proceed to solve for the population in 2015,
or in other words at $t = 5$.

\begin{align}
	\label{p4.3}
	(\ref{g4.1}) \iff \frac{1}{P} \frac{dP}{dt} = A \iff \int \frac{1}{P} dP = \int A dt \\
	\label{p4.4}
	(\ref{p4.3}) \iff \ln P = At + B \textrm{ for some } B \in \reals \\
	(\ref{p4.4}) \iff P(t) = Ce^{At} \textrm{ for } C = e^B \\
	P(0) = 100 \iff C = 100 \\
	P(t) = 100e^{At} \\
	P(2) = 400 \iff 100e^{2A} = 400 \iff e^{2A} = 4 \iff A = \ln 2 \\
	\therefore P(t) = 100 \cdot 2^t \land P(5) = 3200 \textrm{ hamsters or something }
\end{align}

Then, we will have 3200 gerbils in 2015 and the population will double every year thereafter
until the entire universe is consumed by the gerbil population. Doomsday...

\medskip

\textbf{5. Compute the general solution}

Given:

\begin{align}
	\label{g5.1}
	y'' - 2y' + 5y = 0
\end{align}

We guess that $y = e^{rx}$ for some $r \in \cplx$, then:

\begin{align}
	\label{p5.2}
	y' = re^{rx} \land y'' = r^2e^{rx} \\
	\label{p5.3}
	(\ref{g5.1}) \land (\ref{p5.2}) \iff r^2e^{rx} - 2re^{rx} + 5e^{rx} = 0 \\
	\label{p5.4}
	(\ref{p5.3}) \implies r^2 - 2r + 5 = 0 \implies r \in \{ 1-2i, 1+2i \} \\
	y(x) = Ae^{1-2i} + Be^{1+2i} \textrm{ for some } A,B \in \cplx \\
	e^{ix} = \cos(x) + i \sin(x) \iff y(x) = e^x\Big((A+B)\cos(2x) + i(B-A)\sin(2x)\Big) \\
	\therefore y(x) = e^x\Big(C\cos(2x) + D\sin(2x)\Big) \textrm{ for } C = A + B \land D = i(B-A)
\end{align}

\medskip

\textbf{6. Determine linear independence of functions}

Given:

\begin{align}
	f(x) = xe^x \\
	g(x) = x^2
\end{align}

We also have the following derivatives:

\begin{align}
	\frac{df}{dx} & = e^x + xe^x \\
	\frac{dg}{dx} & = 2x
\end{align}

We use the Wronskian determinant to determine linear indepenence.
Two functions $f$ and $g$ are linearly independent
iff their Wronskian determinant $W(f,g) \neq 0$
where $0$ is the zero function in the vector space
of continuous functions.

We compute:

\begin{align}
	W(f,g) = \begin{vmatrix} xe^x & x^2 \\
	e^x+xe^x & 2x \end{vmatrix} =
	2x^2e^x-x^2e^x+x^3x^x \neq 0
\end{align}

We see that $W(f,g) \neq 0$
so we conclude that $f$ and $g$ are linearly independent.

\medskip

\textbf{7. Mass attatched to a spring with a dampener}

Suppose we have an $m = 2$ Kg mass subject
a dampening force with the coefficient $c = 14$
pulled away from a spring $ \Delta x = -2$ meters
at a force of $F = 48$ Newtons.

By Hooke's law, we have $F = -k \Delta x \iff 48 = 2k \iff k = 24$.

Free spring mass movement is described by the following equation
where $x(t)$ is a function of the position of the mass
relative the the equilibrium position of the spring
with respect to time:

\begin{align}
	\label{g7.1}
	mx'' + cx' + kx = 0
\end{align}

We are also given the follwing initial values:

\begin{align}
	x(0) & = 2 \\
	x'(0) & = 3
\end{align}

Now, we can plug in the above values to equation \ref{g7.1} and solve
by guessing that $x(t) = e^{rt}$ for some real $r$:

\begin{align}
	2x'' + 14x' + 24x = 0 \\
	2r^2e^{rt} + 14re^{rt} + 24e^{rt} = 0 \\
	2r^2+ 14r+ 24 = 0 \\
	r^2+ 7r+ 12 = (r+3)(r+4) = 0 \implies r \in \{ -3, -4 \} \\
	x(t) = Ae^{-3t}+Be^{-4t} \textrm{ for some } A,B \in \reals \\
	x(0) = 2 \iff A+B = 2 \\
	x'(t) = -3Ae^{-3t}-4Be^{-3t} \\
	x'(0) = 3 \iff -3A-4B = 3 \\
	\begin{bmatrix} 1 & 1 & 2 \\ -3 & -4 & 3 \end{bmatrix} \sim
	\begin{bmatrix} 1 & 1 & 2 \\ 0 & -1 & 9 \end{bmatrix} \sim
	\begin{bmatrix} 0 & 1 & 11 \\ 0 & -1 & 9 \end{bmatrix} \iff
	A = 11 \land B = -9 \\
	\therefore x(x) = 11e^{-3t}-9e^{-4t}
\end{align}


\medskip

\textbf{Appendix. Original scratch work}

\medskip

\includegraphics[scale=0.075]{1.jpg}

\includegraphics[scale=0.075]{2.jpg}

\includegraphics[scale=0.075]{3.jpg}

\includegraphics[scale=0.075]{4.jpg}

\includegraphics[scale=0.075]{5.jpg}

\includegraphics[scale=0.075]{6.jpg}

\includegraphics[scale=0.075]{7.jpg}

\end{document}
